% PXX.XX Book

\question[2] 
Es posible asociar una entropía $S$ a un agujero negro, la cual da cuenta de la
información de la materia-energía que ha cruzado el horizonte de eventos y que no puede volver a
salir. Esta viene dada por la famosa fórmula de Bekenstein-Hawking

\begin{equation}
  S = \frac{k_B}{L^2_p}\frac{A}{4}, \label{eq:bh-entropy}
\end{equation}
donde $A$ es el área del horizonte de eventos del agujero negro, $k_B$ la
constante de Boltzmann $L_p\equiv\sqrt{\hbar G/c^3}$ es la \textit{longitud de
Planck}, siendo $\hbar$ la constante reducida de Planck, $c$ la velocidad de
la luz en el vacío, y $G$ la constante de gravitación universal. En lo
siguiente, puede utilizar unidades donde $G=c=\hbar=k_B=1$.

\begin{itemize}
  \item[(a)] Calcule el área del horizonte de eventos del agujero negro de
    Schwarzschild en términos de su masa.
    
  \item[(b)] Escriba la entropía del agujero \eqref{eq:bh-entropy} en términos de su masa.
      Muestre que la entropía aumenta al aumentar la masa del agujero.

  \item[(c)] ¿Es este resultado consistente con la segunda ley de la
    Termodinámica? ¿Qué implicancias tiene esto último sobre el área del agujero negro?
\end{itemize}


% \textbf{\droppoints}
\droptotalpoints
