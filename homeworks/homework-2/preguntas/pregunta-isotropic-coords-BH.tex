% PXX.XX Book

\question[2] 
La métrica de Schwarzschild, en términos de las coordenadas estándar de
Schwarzschild $(t,r,\theta,\phi)$, está dada por (en un sistema de unidades donde $G=c=1$)

\begin{equation}
ds^2 = -\left(1-\frac{2M}{r}\right) dt^2 + \frac{dr^2}{1-2M/r} + r^2 d\Omega^2.
  \label{eq:schw-metric}
\end{equation}

Considere ahora la siguiente transformación de coordenadas, la cual re-define la
coordenada radial en términos de la coordenada radial `isotrópica' $R$ como:
\[
r = R\left(1+\frac{M}{2R}\right)^2.
\]

\begin{itemize}
\item[(a)] 
  Muestre explícitamente que, bajo esta transformación de coordenadas, la
  métrica de Schwarzschild toma la forma
\[
ds^2 = -\frac{\left(1-\frac{M}{2R}\right)^2}{\left(1+\frac{M}{2R}\right)^2}dt^2 + \psi^4\left(dR^2 + R^2 d\Omega^2\right).
\]

donde $\psi = \left(1+\frac{M}{2R}\right)$ (denominado `factor conforme').

\item[(b)] ¿Qué sucede con esta métrica en el radio de Schwarzschild $r=2M$?
\end{itemize}



% \textbf{\droppoints}
\droptotalpoints
