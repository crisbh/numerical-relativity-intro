% PXX.XX Book

\question[2] 
En el electromagnetismo, el \textit{tensor de campo electromagnético} se define como:
\[
F_{\mu\nu} = \partial_\mu A_\nu - \partial_\nu A_\mu
\]
donde \( A^\mu = (\phi, \vec{A}) \) es el cuadri-potencial electromagnético.

\begin{itemize}
  \item[(a)] Argumente si \( F_{\mu\nu} \) es simétrico, antisimétrico, o
    ninguno de estos.

  \item[(b)] Exprese las componentes del tensor \( F_{\mu\nu} \) en términos de
    los campos eléctricos y magnéticos. Específicamente, muestre que:
  \[
  F_{0i} = -E_i, \quad F_{ij} = -\epsilon_{ijk} B^k
  \]
   donde $\epsilon_{ijk}$ es el símbolo de Levi-Civita.

  \item[(c)] Considere un potencial escalar nulo \( \phi = 0 \), y un potencial vector dado por:
  \[
  \vec{A} = \left(0, \frac{B_0 x}{2}, -\frac{B_0 y}{2} \right)
  \]
  donde \( B_0 \) es constante. Calcule las componentes no nulas del tensor \( F_{\mu\nu} \) en este caso.

  \item[(d)] Usando la expresión:
  \[
  B^i = \frac{1}{2} \epsilon^{ijk} F_{jk}
  \]
  calcule explícitamente el campo magnético \( \vec{B} \). Verifique que su
  resultado coincide con la expresión usual:
  \[
  \vec{B} = \nabla \times \vec{A}
  \]
\end{itemize}


% \textbf{\droppoints}
\droptotalpoints
